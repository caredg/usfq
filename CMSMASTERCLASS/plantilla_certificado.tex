\documentclass{letter}[14pt]
\usepackage{url}
\usepackage{marvosym}
%%%%%%%%%%%%%%spanish
\usepackage[spanish]{babel}
\selectlanguage{spanish}
\decimalpoint
%
\usepackage[utf8]{inputenc}
%\usepackage[utf8]{inputenc}
%%%%%%%hanle list in csv
%\usepackage{datatool}
% Load database 'names' from file 'namelist.csv'
%\DTLloaddb{names}{lista_certificados.csv}
%%%%%%%%%For background image frame
\usepackage{wallpaper}
%%%%%%%%%%logos in header and footer
\usepackage{graphicx}
\usepackage{fancyhdr}
\pagestyle{empty}
%stupid geometry
\usepackage{geometry}
\geometry{headheight = 0.3in}
%%%%%%%no lines in header
\renewcommand{\headrulewidth}{0pt}
%%%%%%% no lines in footer
\renewcommand{\footrulewidth}{0pt}
\headsep = 3cm
\textheight = 615pt
\voffset=0.5cm
%%%%modify page style empty to place logos in header
\fancypagestyle{empty}{\fancyhf{}\fancyhead[C]{\includegraphics[height=1.8in,
    keepaspectratio=true]{usfq}}}


%\fancyfoot[L]{\includegraphics[width=0.8in]{claf}}\fancyfoot[C]{\includegraphics[width=0.8in]{epn}}\fancyfoot[R]{\includegraphics[width=0.8in]{espoch}}}\


\signature{
\footnotesize
\begin{tabular}{ccccc}
\includegraphics[width=0.3\textwidth]{signature.jpg}&\hspace{1.5cm}&&&\includegraphics[width=0.3\textwidth]{FIRMAANDREAAYALA_a.png}\\
\hspace{2cm}Edgar Carrera, Ph.D.&\hspace{1.5cm}&&&Andrea Ayala, M.S.\\
\hspace{2cm}\Letter{\ ecarrera@usfq.edu.ec}&\hspace{1.5cm}&&&\Letter{\ aayala@usfq.edu.ec}\\
\hspace{2cm}Mentor/Instructor &\hspace{1.5cm}&&&Instructora\\
\hspace{2cm}Profesor Investigador de Física&\hspace{1.5cm}&&& Profesora de Matemáticas\\
\hspace{2cm}Universidad San Francisco de Quito&\hspace{1.5cm}&&&Universidad
San Francisco de Quitob\\
\end{tabular}
}
%\address{Universidad San Francisco de Quito\\Cumbayá, Diego de Robles y Vía
%  Interoceánica\\Quito, Ecuador\\P.O.BOX:	17-1200-841\\\Telefon{(+593) 2
%    297-1700} }

\address{\ }

\date{Quito, 5 de marzo 2021}
%%to move signature to left
\longindentation=0pt
\begin{document}

\CenterWallPaper{0.8}{cms_inverted_colorized.png}

\begin{letter}{A Quien Interese:}

\opening{\vspace{0.5cm}Certificamos que\\}

{\centering \Large \bf [Nombre]\\}
\vspace{0.5cm}
participó en el {\bf CMS Masterclass Quito 2021} organizado por la Red Quarknet y la Universidad San Francisco de Quito (USFQ).  El evento
tuvo lugar en el campus virtual de la USFQ, en Quito-Ecuador,
durante el miércoles 3 y jueves 4 de marzo de 2021.

Este es un evento realizado en conjunto con el laboratorio CERN, Fermilab y diferentes
universidades a nivel mundial.  Está orientado a dar a conocer a los
estudiantes de colegio el trabajo que se realiza en estos centros de investigación. El
International Masterclass, del cual el CMS Masterclass es parte, es organizado por TU Dresden y QuarkNet Notre Dame en
cooperación con el IPPOG (International Particle Physics Outreach Group), con el fin de promover la ciencia.

El CMS Masterclass tuvo una duración de 4 horas.  A continuación se detallan
las actividades realizadas:

\begin{itemize}
\item Charla sobre el experimento CMS, LHC y CERN (30 min)
\item Uso del laboratorio en línea i-Spy (1 hora)
\item Análisis de datos del experimento CMS utilizando i-Spy y CIMA(1 hora)
\item Discusión de resultados (30 min)
\item Presentación de resultados en videoconferencia con el laboratorio Fermilab (1 hora)
\end{itemize}



%\vspace{1.3cm}
\closing{Atentamente,}
%\vspace{-1.5cm}
%\ps{P.S. Here goes your ps.}
%\encl{Enclosures.}
\end{letter}
\end{document}
